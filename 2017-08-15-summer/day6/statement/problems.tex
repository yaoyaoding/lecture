\documentclass[11pt,a4paper,oneside]{article}
\usepackage[english]{babel}
\usepackage{olymp}
\usepackage[dvips]{graphicx}
\usepackage{color}
\usepackage{colortbl}
%\usepackage{expdlist}
%\usepackage{mfpic}
%\usepackage{comment}
\usepackage{multirow}

\usepackage{xeCJK}

%\setCJKmainfont[BoldFont={Hei}]
%{SimSun}
%\setCJKmonofont{FangSong}

\renewcommand{\contestname}{
No.7 High School Summer Training \\
idy002, \today
}    

\begin{document}

\begin{problem}{fight}{fight.in}{fight.out}{1 second} 

    Mr.Hu来到了一个国家。
    
    这个国家正在招兵,现在有$n$个人准备参加考核。每个人都有三个属性:$a$、$d$和$h$,分别表示其攻击力、防御力和身高。
    
    这个国家有$m$个将军,每个人对于能力的看重不同,有的人觉得攻击力越高越好,有的则觉得防御力更为重要。具体来说,第$i$个将军会对攻击力和防御力的看重程度分别为$p$和$q$,那么他认为一个攻击力和防御力分别为$a$和$d$的人的战斗力是
    $$
	    f = pa + qd
    $$
    每个将军还有一个战斗力要求$r$,只有他认为一个士兵的战斗力大于等于$r$时,他才会招收该士兵。
    
    现在,国家元首想问你,如果第$i$位将军负责招收士兵,那么不被招收的人的身高和是多大?
    
    \InputFile

	第$1$行包含两个整数:$n \; m$,表示参加考核的士兵数和将军人数。
	
	接下来$n$行,每行三个整数:$a \; d \; h$。
	
	接下来$m$行,每行三个整数:$p \; q \; r$。
 
    \OutputFile

	输出$m$行,每行一个整数,表示身高和。
  
    \Example

    \begin{example}
        \exmp{
		     3 3
		     1 2 5
		     3 1 4
		     2 2 1
		     2 1 6
		     1 3 5
		     1 3 7
        }{
			 5
			 0
			 4
        }%
    \end{example}

    \Note
    
    \begin{itemize}
		\item 对于$30\%$的数据,$1 \leq n, m \leq 5000$。
		\item 对于$100\%$的数据,$1 \leq n, m \leq 50000$,$150 \leq h \leq 200$,$0 \leq a, d, p, q, r \leq 10^9$。
    \end{itemize}

\end{problem}

\begin{problem}{surround}{surround.in}{surround.out}{1 second} 
	
	Mr.Hu现在在指挥一场战争!
	
	现在我们有$n$个部队在战场上,敌军有$m$个部队在战场上,每个部队所在的位置都可以用一个二维坐标来表示(我方部队有可能正在和敌方部队交战,所以坐标有可能重合)。
	
	现在我们急需抽调出一批部队到另一战场上去,但Mr.Hu又不想放弃歼灭敌人的机会,所以,他想留下最少的部队,使得这些部队能够将敌军所有部队包围(即我方部队所组成的凸包包含敌军所有部队,在边界上也算包围)。
	
	作为Mr.Hu的参谋长,你需要为Mr.Hu计算出最小需要留下的部队数量。
	
	\InputFile
	
	第$1$行两个整数:$n \; m$,分别表示我方部队数和敌军部队数。
	
	接下来$n$行,每行两个数:$x \; y$,表示我方部队的位置。
	
	接下来$m$行,每行两个数:$x \; y$,表示敌方部队的位置。
	
	\OutputFile
	
	输出一行,包含一个整数,表示最小需要留下的部队数,如果我方无论如何都不能全歼敌军,那么输出-1。
	
	\Example
	
	\begin{example}
		\exmp{
			10 10
			0.1 0.2
			0.2 0.3
			0.3 0.4
			0.4 0.5
			0.5 0.1
			0.6 0.2
			0.7 0.3
			0.8 0.1 
			0.9 0.1 
			1 0 
			0.1 0.2 
			0.2 0.3 
			0.3 0.4 
			0.4 0.5 
			0.5 0.1 
			0.6 0.2 
			0.7 0.3
			0.8 0.1 
			0.9 0.1
			1 0 
		}{
			5
		}%
\end{example}

\Note

\begin{itemize}
	\item 对于$30\%$的数据,$1 \leq n, m \leq 12$.
	\item 对于$100\%$的数据,$1 \leq n, m \leq 500$,$0.0 \leq x, y \leq 1.0$
\end{itemize}

\end{problem}

\begin{problem}{count}{count.in}{count.out}{1 second} 
	
	Mr.Hu觉得大家很久没有数数了,所以决定让大家练习一下数数。
	
	现在Mr.Hu给出一个简单无向加权图(且无重边,无自环),你不满足于求这个图的一个最小生成树,也不满足于统计这个图的生成树个数,于是,你决定数一数最小生成树的个数。
	
	\InputFile
	
	第$1$行两个整数:$n \; m$,表示图的点数和边数。
	
	接下来$m$行,每行三个整数:$a \; b \; c$,表示$a$与$b$之间有一条边权为$c$的边。
	
	\OutputFile
	
	输出$1$行,表示最小生成树的个数模$31011$.
	
	\Example
	
	\begin{example}
		\exmp{
			4 6
			1 2 1
			1 3 1
			1 4 1
			2 3 2
			2 4 1
			3 4 1
		}{
			8
		}%
	\end{example}
	
	\Note
	
	\begin{itemize}
		\item 对于$30\%$的数据,$1 \leq n, m \leq 12$。
		\item 对于$100\%$的数据,$1 \leq n \leq 100$,$1 \leq m \leq 1000$,$1 \leq c \leq 10^6$,保证同一个边权最多出现$10$次。
	\end{itemize}

\end{problem}


\end{document}
