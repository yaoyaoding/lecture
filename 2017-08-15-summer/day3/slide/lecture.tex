\documentclass[9pt]{beamer}
\usepackage{xeCJK}

\usetheme{Berkeley}
\usefonttheme[onlymath]{serif}
\usepackage[utf8]{inputenc}
\usepackage[T1]{fontenc}
\usepackage{amsmath}
\usepackage{amsfonts}
\usepackage{amssymb}
\usepackage{multicol}

\title{数学第三讲}
\subtitle{高斯消元、快速傅里叶变换}
\author{丁尧尧}
\institute{上海交通大学}
\date{\today}
\usetheme{CambridgeUS}

\begin{document}
	\maketitle
	\begin{frame}{目录}
		\tableofcontents
	\end{frame}
	
	\section{高斯消元}
		\subsection{解方程}
		\begin{frame}
			求解下列方程组:
			\begin{align*}
				a_{11}x_1 + a_{12}x_2 + \cdots + a_{1n}x_n & =  c_1 \\
				a_{21}x_1 + a_{22}x_2 + \cdots + a_{2n}x_n & =  c_2 \\
				a_{31}x_1 + a_{32}x_2 + \cdots + a_{3n}x_n & =  c_3 \\
				\cdots & \\
				a_{m1}x_1 + a_{m2}x_2 + \cdots + a_{mn}x_n & =  c_m 
			\end{align*}
			
			\pause
			怎么解?
		\end{frame}
		\subsection{方阵的行列式}
		\begin{frame}
			\begin{itemize}
			\item \pause 行列式是什么?	\\
				\pause 一种定义是:
				$$
					det(A) = \sum_{i_1i_2\dots i_n}(-1)^{\sigma(i_1i_2\dots i_n)}a_{1i_1}a_{2i_2}\dots a_{ni_n}
				$$
				其中$i_1i_2\dots i_n$是一个$1$到$n$的排列,$\sigma(i_1i_2\dots i_n)$表示该排列的逆序对数。
			\item \pause 行列式的几何意义。
			\item \pause 行列式的性质 \\
				\pause 
				\begin{itemize}
					\item \pause 某行乘一个数,那么行列式也乘那个数。
					\item \pause 某行加到另一行,行列式不变
					\item \pause 交换两行,行列式取反相反数
				\end{itemize}
			\item \pause 行列式的计算。\\
				  \pause 高斯消元!

			\end{itemize}
		\end{frame}
		
		\subsection{矩阵求逆}
		\begin{frame}{矩阵乘法}
			矩阵乘法大家都知道:
			$$
				A_{n\times m}B_{m\times r} = C_{n\times r}
			$$
			对于方阵而言,假如:
			$$
				AB = BA = E
			$$
			那么$A$、$B$互为逆矩阵。
			
			有两个问题:
			\begin{itemize}
				\item 什么时候一个方阵存在逆矩阵?
				\item 如果存在,怎么求?
			\end{itemize}
		\end{frame}
		\begin{frame}
			对于第一个问题:一个矩阵存在逆矩阵当且仅当其满秩(秩是行向量组或列向量组的最大线性无关组的大小)当且仅当矩阵行列式非0.
			
			我们怎么求一个方阵的逆矩阵呢?
			\begin{enumerate}
				\item 矩阵进行如下操作可以相当于用一个矩阵乘以它:
				
				\begin{itemize}
					\item 将一行上的所有数乘以$k$
					\item 将一行加到另一行上
					\item 交换两行
				\end{itemize}
				
				\item 求逆的过程
				
				如果要求矩阵$A$的逆矩阵$A^{-1}$,先得到一个单位矩阵$B$,
				然后用上面1中的三种操作将$A$变成单位矩阵(不能变成单位矩阵则说明该矩阵行列式为0,即该矩阵不存在逆)
				
				将对$A$的所有操作同样地应用于$B$,最终$B$就是$A^{-1}$
				\item  求逆的正确性
				 我们对$A$进行了一系列变换,等同于用一个矩阵$C$乘以$A$使得 $CA = I$,即$C$是$A$的逆矩阵, 将同样的操作作用于$B$,得到的矩阵为 $CB = CI = C$,即最终B的结果就是我们要求的逆
			\end{enumerate}
		\end{frame}
		\subsection{Matrix-Tree定理}
		\begin{frame}
			一些概念:\pause 
			\begin{itemize}
				\item 度数矩阵 \pause 
				\item 邻接矩阵 \pause 
				\item 基尔霍夫矩阵 = 度数矩阵 - 邻接矩阵
			\end{itemize}
			
			\pause
			Matrix-Tree定理: \pause
			\begin{theorem}
				一个$n$个点$m$条边的无向图的生成树总数为其对应的基尔霍夫矩阵的$n-1$阶余子式。
			\end{theorem}
		\end{frame}
		
	\section{快速傅里叶变换}
		\subsection{快速傅里叶变换}
		\begin{frame}
			\begin{itemize}
				\item \pause 快速傅里叶算法解决的问题:快速进行多项式乘法(卷积)。
				\item \pause 多项式的两种表示方法:系数表示法和点值表示法
				\item \pause 点值表示法的优点
				\item \pause 复数域$n$次单位根
				\item \pause 快速地将系数表示法转化成点值表示法
				\item \pause 快速地将点值表示法转换成系数表示法
				\item \pause 常数更小的写法? \pause 非递归写法!
				\item \pause 一定要在复数域? \pause 对于一些特殊的模空间也可以做。 
				\item \pause 一般的模空间 ?
			\end{itemize}
		\end{frame}
		\subsection{一些题目}
		\begin{frame}
		\end{frame}
	
\end{document} 

