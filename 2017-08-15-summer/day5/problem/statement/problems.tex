\documentclass[11pt,a4paper,oneside]{article}
\usepackage[english]{babel}
\usepackage{olymp}
\usepackage[dvips]{graphicx}
\usepackage{color}
\usepackage{colortbl}
%\usepackage{expdlist}
%\usepackage{mfpic}
%\usepackage{comment}
\usepackage{multirow}

\usepackage{xeCJK}

%\setCJKmainfont[BoldFont={Hei}]
%{SimSun}
%\setCJKmonofont{FangSong}

\renewcommand{\contestname}{
No.7 High School Summer Training \\
idy002, \today
}    

\begin{document}

\begin{problem}{near}{near.in}{near.out}{1 second} 

	 Mr.Hu最近去摘果子,这个果园里面有$n$棵果树,在不同的位置,但有些果子成熟了,有些则没有,Mr.Hu一共去了$m$次果园,每次去他都会被告知,某个简单多边形区域内的果子是成熟的(边界上也算成熟),可以去采摘,Mr.Hu只会去离他最近的两棵果树摘果子,现在Mr.Hu想考考你,希望你能够帮他把计算出他某次去果园时,如果他在某个给定的位置,它会去哪两棵树摘果子。
 
    \InputFile
    
    第$1$行,一个整数$T$,表示数据组数,
	
	每组数据第$1$行,一个整数$n$,表示果树的数量。
	
	接下来$n$行,每行两个整数:$x \; y$,表示第$i$棵果树的位置。
	
	接下来$1$行,一个整数$m$,表示Mr.Hu去果园的次数。
	
	接下来$m$个部分。
	
	每个部分的第$1$行表示本次去,成熟的果子所在的简单多边形的点数$B$。
	
	接下来$B$行,每行两个整数:$x \; y$,表示点的坐标(按顺时针或逆时针给出)。
	
	接下来$1$行,一个整数$q$想问你的问题个数。
	
	接下来$q$行,每行两个整数:$x \; y$,表示询问的点的位置。

    \OutputFile

	对每组数据,输出包含$m$个部分,每个部分包含其对应的$q$行,每行两个整数:$a \; b$,表示最近的两个点的标号(按距离为第一关键字,标号为第二关键字,从小到大排序)。
 
    \Example

    \begin{example}
        \exmp{
        	1
        	8
        	3 3
        	2 15
        	7 15
        	5 4
        	13 9
        	11 10
        	2 8
        	10 4
        	2
        	4
        	1 1
        	1 16
        	16 16
        	14 1
        	2
        	5 5
        	7 6
        	4
        	3 15
        	16 15
        	13 11
        	15 7
        	3
        	9 9
        	4 13
        	15 9
         }{
	         4 1
	         4 8
	         6 5
	         3 6
	         5 6
        }%
    \end{example}

    \Note
    
    \begin{itemize}
		\item 对于$10\%$的数据,$1 \leq n \leq 1000$,$1 \leq q \leq 100$,给出的区域包含所有果树。
		\item 对于$30\%$的数据,$1 \leq n \leq 1000$,$1 \leq q \leq 100$,给出的区域是一个凸包。
		\item 对于$50\%$的数据,$1 \leq n \leq 1000$,$1 \leq q \leq 100$。
		\item 对于$100\%$的数据,$1 \leq n \leq 10000$,$1 \leq q \leq 3000$,$1 \leq m \leq 10$, $1 \leq B \leq 20$,$-10^7 \leq x, y \leq 10^7$, $\sum n \leq 20000$,保证区域至少包含两棵果树。
    \end{itemize}

\end{problem}

\begin{problem}{fish}{fish.in}{fish.out}{1 second} 
	
	Mr.Hu 最近在和小伙伴们捕鱼,他们一共有四个人,一共有$n$个可供站立的位置,现在Mr.Hu想知道怎样安排人的站立,使得他们围成的区域的面积最大,因为这样才能捕到最多的鱼(两个人可以在同一个点)。
	
	\InputFile
	
	第$1$行,一个整数$n$,表示点的个数。
	
	接下来$n$行,每行$2$个数:$x \; y$,表示点的坐标。
	
	\OutputFile
	
	输出一行,包含一个数,表示面积的最大值,保留三位小数。
	
	\Example
	
	\begin{example}
		\exmp{
			5	
			0 0	
			1 0	
			1 1	
			0 1		
			0.5 0.5
		}{
			1.000
	}%
\end{example}

\Note

\begin{itemize}
	\item 对于$30\%$的数据,$1 \leq n \leq 50$。
	\item 对于$50\%$的数据,$1 \leq n \leq 200$。
	\item 对于$100\%$的数据,$1 \leq n \leq 2000$,所有数绝对值不超过$1000$。
\end{itemize}

\end{problem}

\begin{problem}{area}{area.in}{area.out}{1 second} 
	
	Mr.Hu最近有些无聊,就在纸上画了$n$个开口向上的抛物线,并且这些抛物线与$x$轴至多有一个交点。通过这些函数,我们可以构造一个新的函数:
	$$
		g(x) = \min_{i = 1}^{n}f_i(x)
	$$
	
	现在,Mr.Hu想问你在$x \in [L,R]$这个范围内,$g(x)$与$x$轴围成的面积是多少。
	
	\InputFile
	
	第$1$行,一个整数$n \; q$,表示抛物线条数,和询问次数。
	
	接下来$n$行,每行$3$个数:$a \; b \; c$,表示$f_i(x) = ax^2 + bx + c$。
	
	接下来$q$行,每行两个数:$L \; R$,表示一次询问。
	
	\OutputFile
	
	输出一行,包含一个数,表示面积,保留三位小数。
	
	\Example
	
	\begin{example}
		\exmp{
			2
		}{
			1.000
		}%
\end{example}

\Note

\begin{itemize}
	\item 对于$30\%$的数据,$n = 1$。
	\item 对于$50\%$的数据,$n = 2$。
	\item 对于$100\%$的数据,$1 \leq n \leq 50$,$1 \leq q \leq 10$, $L \leq R$, 其他所有数的绝对值不超过$50$。
\end{itemize}

\end{problem}

\end{document}
