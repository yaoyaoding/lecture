\documentclass[9pt]{beamer}
\usepackage{xeCJK}

\usetheme{Berkeley}
\usefonttheme[onlymath]{serif}
\usepackage[utf8]{inputenc}
\usepackage[T1]{fontenc}
\usepackage{amsmath}
\usepackage{amsfonts}
\usepackage{amssymb}
\usepackage{multicol}


\title{数学第二讲}
\subtitle{离散对数、元根、反演}
\author{丁尧尧}
\institute{上海交通大学}
\date{\today}
\usetheme{PaloAlto}

\begin{document}
	\maketitle
	\begin{frame}{目录}
		\tableofcontents
	\end{frame}
	
	\section{离散对数}
		\subsection{离散对数定义}
		\begin{frame}
			对于以下问题:
			\begin{definition}[离散对数]
				给定$a$,$b$,$m$,其中$a$与$m$互素,求最小的非负(正)整数$x$,使得:
				$$
				a^x \equiv b \quad (mod \; m)
				$$
				我们称$x$为在模$m$意义下,以$a$为底的$b$的离散对数,记作$ind_ab$。
			\end{definition}
			
			\pause
			
			给出$a,b,m$,(我们假设$a$与$m$互素)我们如何求$x$呢?
		\end{frame}
		\subsection{大步小步走}
		\begin{frame}
			有一个叫做大步小步的算法(Baby Step Giant Step)。
			我们假设$x = ic + j$是一个答案(其中$c$是我们自己选定的一个在$2,m-1$之间的数)。
			我们先计算出:
			$$
			a^0, a^1, a^2, a^3, \cdots, a^{c-1}
			$$			
			如果发现其中某个值是$b$,那么我们就找到答案了。
			
									\pause
									
			否则,我们将这些数放在一个数据结构中(平衡二叉树,哈希表都可以),要求可以通过$a^i$的值快速得到$i$。那么我们再依次算出:
			$$
			ba^{-c}, ba^{-2c}, \cdots, ba^{-kc}
			$$
			
						\pause
			
			每算完一个$ba^{-ic}$,我们就看上面的数据结构中是否有一个值$a^j$等于它,如果有,那么它们满足:
			$$
			a^j \equiv ba^{-ic} \quad (mod \; m)
			$$
			即:
			$$
			a^{ic + j} \equiv b \quad (mod \; m)
			$$
		\end{frame}
		\begin{frame}
			分析复杂度,如果我们上面用哈希表存,那么可以$O(1)$判断某个值是否存在。
			
			那么我们总共需要计算的数的个数是$O(b + \frac{m}{b})$,我们取$b = \sqrt{m}$,
			
			可以得到$O(\sqrt{m})$的复杂度.
			
						\pause
			
			上面的方法,$a$不限于整数,还可以是矩阵,但都有同一个要求,即$a$存在乘法逆元.(有方法可以避免求逆元,但是还是要求逆元存在。)
		\end{frame}
	\section{元根}
		\subsection{一些概念}
		\begin{frame}
			我们先介绍一些概念.
			\begin{definition}[剩余系]
				对于给定模数$m(m > 0)$,如果有一组数$\{ a_i\}$:
				$$
					a_1, a_2, a_3,\dots, a_{m}
				$$
				对于任何数$a$,存在唯一数$a_i$满足:
				$$
					a \equiv a_i \quad (mod \; m)
				$$
				那么我们将$\{a_i\}$称作模$m$的一组完全剩余系.
			\end{definition}
			
		\end{frame}
		\begin{frame}
			类似的有:
			\begin{definition}[既约剩余系]
				对于给定模数$m(m > 0)$,如果有一组数$\{ a_i\}$:
				$$
					a_1, a_2, a_3,\dots, a_{k}
				$$
				满足:
				$$
					gcd(a_i,m) = 1
				$$
				且对于任何和$m$互质的数$a$,有唯一的$a_i$满足:
				$$
					a \equiv a_i \quad (mod \; m)
				$$
				那么我们将$\{a_i\}$称作模$m$的一组既约剩余系.
			\end{definition}
			
						\pause
			
			模$m$的既约剩余系的个数记作$\varphi(m)$.
		\end{frame}
		\subsection{阶和元根}
		\begin{frame}
			\begin{definition}[阶]
				给定一个与$m(1 \leq m)$互素的$a$,则最小的一个满足:
				$$
					a^r \equiv 1 \quad (mod m)
				$$
				的正整数$r$叫做$a$模$m$的阶,一般记作$r = \delta_m(a)$
			\end{definition}
			
						\pause
			
			\begin{definition}[元根] 
				对于模数$m$,如果存在一个数$g$,满足:
				$$
					\delta_m(g) = \varphi(m)
				$$
				我们则称$g$为模$m$的一个元根
			\end{definition}
		\end{frame}
		\begin{frame}
			我们知道,如果$a$与$m$互素,那么:
			$$
				a^1, a^2, a^3,\dots,a^i,\dots
			$$
			都与$m$互素,即它们都是缩系的元素.
			
						\pause
			
			元根的意义在于,将缩系中的每一个元素,都与一个$g^i$这种形式对应起来.
			
			假如我们找到了一个模$m$的元根$g$,想求其缩系中的一个元素$a$对应的指数是什么,我们就可以用离散对数找到满足:
			$$
				g^i \equiv a \quad (mod m)
			$$
			的$i$.
		\end{frame}
		\begin{frame}
			并不是所有数都有元根.
			\begin{theorem}
				只有形如:
				$$
				1,2,4,p^{\alpha},2p^{\alpha}
				$$
				的数存在元根(其中$\alpha \geq 1$且$p$是奇素数).
			\end{theorem}
			
						\pause
			
			那么我们怎样找元根呢?
			
						\pause
			
			\begin{theorem}
				如果存在正数$a,b,m$,且$gcd(a,m) = 1$, 满足
				$$
				a^b \equiv 1 \quad (mod m)
				$$
				那么有:
				$$
				\delta_m(a) \mid b
				$$
			\end{theorem}
			
		\end{frame}
		\begin{frame}
			通过上面这个定理可以证明:
			\begin{theorem}
				对于给定的与$m \geq 2$互素的一个数$g$, $g$是$m$的一个元根当且仅当对于$\varphi(m)$的所有素因子$p_i$,有:
				$$
				g^{\frac{\varphi(m)}{p_i}} \not \equiv 1 \quad (mod \; m)
				$$
			\end{theorem}
			
						\pause
			
			因为在$10^9$范围内的所有素数的最小的元根都很小(最大的不过一百多),所以我们可以暴力从小到大check.
		\end{frame}
		\subsection{例题}
		\begin{frame}
			我们来道例题看看:
			\begin{block}{例题1}
				给你$a,b,m$,都是正整数,其中$m$是素数,求:
				$$
					a^x \equiv b \quad (mod \; m)
				$$
				其中:$2 \leq m \leq 2\times10^9$,$1 \leq a, b\ < m$
			\end{block}
			
						\pause
			
			很容易离散对数就可以秒了对不对.
		\end{frame}
		\begin{frame}
			\begin{block}{例题2}
				给你$a_1,b_1,a_2,b_2,m$,都是正整数,其中$m$是素数,求满足下面条件的$x$:
				$$
					a_i^x \equiv b_i \quad (mod\; m) \quad(i = 1, 2)
				$$
				其中:$2 \leq m \leq 2\times10^9$且$1 \leq a_i, b_i < m$.
			\end{block}

		\end{frame}
		\begin{frame}
				先找到$m$的一个元根,然后找到$a_i$和$b_i$的离散对数:
				$$
				g^{c_i} \equiv a_i \quad (mod \; m)
				$$
				$$
				g^{d_i} \equiv b_i \quad( mod \; m)
				$$
				
							\pause
				
				然后就把问题化简成了:
				$$
				g^{xc_i} \equiv g^{d_i} \quad( mod \; m)
				$$
				因为$g$是模$m$的元根,所以上面的方程等价于:
				$$
					xc_i \equiv d_i \quad(mod \quad m - 1)
				$$
				
							\pause
				
				从而把问题转化为解一元一次同余方程组的问题.
			\end{frame}
						
			\begin{frame}
				\begin{block}{例题3}
				给你三个正整数$a,b,m$,其中$m$是质数,求$x$满足:
				$$
						x^a \equiv b \quad (mod\;  m)
				$$
				其中:$1 \leq x, b< m$
				\end{block}
				
							\pause
				
				同样先求离散对数,然后解一次方程,得到$ind_g(x)$,最后快速幂一下就行了.
			\end{frame}
			\begin{frame}
				元根,离散对数,主要的作用是把一些和指数有关的问题转化成一般的一次同余方程,类似于正实数上的开$log$运算.
				
				只是在模意义下,我们的底需要精细地选取.
			\end{frame}
	
		\section{反演}
			\subsection{引入}
			\begin{frame}
				\begin{definition}[数论函数]
					定义域是正整数,值域为复数域的函数是数论函数。
				\end{definition}
				有这样一个问题:
				\begin{block}{问题}
					存在一个数论函数$f(n)$,由它可以产生一个数论函数:
					$$
						F(n) = \sum_{d \mid n} f(d)
					$$
					假如我们知道$F(n)$,怎样求得$f(n)$呢?
				\end{block}
				请大家先自行解决这个问题(提示:容斥原理)
			\end{frame}
			\begin{frame}
				因为我们要的是一个普适的规律,所以我们把$f(i)$看成一个个独立的元素,而把$F(n)$看成某些$f(i)$构成的集合。(比如:$F(9) = \{f(1), f(3), f(9)\}$)。
				
				我们发现,$F(n)$中包含了$f(n)$,所以我们可以尝试用把$F(n)$中多加的元素去掉,来得到$f(n)$。
				
								\pause
				
				我们来看看$F(n)$包含了哪些元素。
				
				设$n = p_1^{a_1}p_2^{a_2}\cdots p_k^{a_k}$,那么$F(n)$包含的元素就是
				$$
				F(n) = \{f(d) \mid d = p_1^{b_1}p_2^{b_2}\cdots p_k^{b_k}, 0 \leq b_i \leq a_i\}
				$$
				
							\pause
				
				我们的目标是得到$\{f(n)\}$(不妨用$f(n)$来表示这个集合)。
				
			\end{frame}
			
			\begin{frame}
				
				考虑用容斥原理来解决这个问题。思考后我们发现,有:
				$$
					f(n) = F(n) - \bigcup_{p_i \mid n}F(\frac{n}{p_i})
				$$
				
							\pause
				
				用容斥原理,我们可以把后面那个求并打开(注意,下面我们把集合理解成可重集,运算$+$就是把两个集合的元素放在一起,不去重,运算$-$指去掉后面那个集合的元素,后面有几个去几个,交和并还是要去重):
				$$
					f(n) = F(n) - F(\frac{n}{p_1}) - F(\frac{n}{p_2}) - \cdots + F(\frac{n}{p_1})\cap F(\frac{n}{p_2}) \cdots - \cdots
				$$
				
							\pause
				
				我们有:
				$$
					F(\frac{n}{p_1})\cap F(\frac{n}{p_2}) = F(\frac{n}{p_1p_2})
				$$
				替换后,我们两边求个和(下面就是代表的数值而不是集合了):
				$$
					f(n) = F(n) - F(\frac{n}{p_1}) - \cdots + F(\frac{n}{p_1p_2}) + \cdots - F(\frac{n}{p_1p_2p_3}) - \cdots 
				$$
				
				
			\end{frame}
			
			\begin{frame}
				我们发现,$f(n)$可以用一些$F(d)$来表示,且$d \mid n$。
				
				如果$\frac{n}{d}$不是一些素数单个乘起来,那么$F(d)$就不出现在我们的结果里。
				
				否则,$\frac{n}{d}$的素因子个数决定了$F(d)$前的系数,奇数个为负,偶数个为正。
				
							\pause
				
				举个例子:$f(12) = F(12) - F(6) - F(4) + F(2)$。
				
			\end{frame}
			
			\subsection{莫比乌斯函数}
			\begin{frame}{莫比乌斯函数}
				\begin{definition}[莫比乌斯函数]
					假如正整数有如下质因数分解:$n = p_1^{a_1}p_2^{a_2}\cdots p_k^{a_k}$,那么有:
					$$
						\mu(n) = 
							\begin{cases}
								1 & n = 1 \\
								(-1)^k & a_i = 1 \\
								0 & \text{某个$a_i>1$} 
							\end{cases}
					$$
				\end{definition}
				
							\pause
				
				我们发现,这个函数就是我们上面所说的系数。通过这个函数,我们就可以把我们上面得到的结果表示成:
				$$
					f(n) = \sum_{d \mid n} F(d)\mu(\frac{n}{d}) = \sum_{d \mid n} F(\frac{n}{d})\mu(d)
				$$
			\end{frame}
			
			\begin{frame}
				由上面的讨论,我们发现莫比乌斯函数本质就是处理关于"质因子"容斥时的系数,所以它的用处不光用于反演。
				
							\pause
				
				\begin{block}{思考}
					现在有一些范围在$[1,n]$中的整数,已知他们中是$d$的倍数的数有$F(d)$个,求$f(d)$,表示等于$d$的个数。
					
					给你$F(1),F(2),\cdots,F(n)$,
					
					需要你求$f(1), f(2), \cdots, f(n)$。(假如你有能力$O(n)$求出$\mu(1), \mu(2), \cdots , \mu(n)$)。
				\end{block}
				
				\pause
				有$O(nlogn)$做法吗?
				
				\pause
				有$O(n)$做法吗?
			\end{frame}
			
			\begin{frame}
				莫比乌斯函数有一个很重要的性质:
				$$
					\sum_{d \mid n} \mu(d) = [n == 1]
				$$
				其中$[n == 1]$是一个关于$n$的函数,$n$为$1$时它是$1$,否则它为$0$。
				
				\pause
				证明很简单,大家思考一下。
				
				\pause
				这个公式是我们推反演的时候经常使用的。
			\end{frame}
			
			\subsection{莫比乌斯反演}
			\begin{frame}{莫比乌斯反演}
				我们进入正题。
				\begin{theorem}[莫比乌斯反演公式]
					设$f(n)$为一个数论函数,我们定义:
					$$
						F(n) = \sum_{d \mid n} f(d)
					$$
					那么有:
					$$
						f(n) = \sum_{d \mid n} F(n)\mu(\frac{n}{d})
					$$
				\end{theorem}

				
			\end{frame}
			
			\begin{frame}
				这个公式我们上面已经证明过了。但运用上面那个莫比乌斯函数的性质有个更简洁的证明:
				
				\begin{align}
					\sum_{d \mid n} \mu(d) F(\frac{n}{d}) & = \sum_{d \mid n} \mu(d) \sum_{c \mid \frac{n}{d}} f(c)  \\
					& = \sum_{c \mid n} f(c) \sum_{d \mid \frac{n}{c}} \mu(d) \\
					& = \sum_{c \mid n} f(c) [n == c] \\
					& = f(n)
				\end{align}
				证毕。
				
							\pause
				
				注意,上面那两个公式本质是可以互推的,上推下已经完成,请自己下来推一推下推上。
			\end{frame}
			
			\begin{frame}{另一种形式}
				莫比乌斯反演还有一种形式:
				$$
					f(n) = \sum_{n \mid d} F(d)\mu(\frac{d}{n})
				$$
				
							\pause
				
				证明和上面类似,请自己下来推一推。				
			\end{frame}
			
			\begin{frame}{积性函数}
				\begin{definition}[积性函数]
					一个数论函数$f(n)$是积性的,当且仅当:
					$$
						f(ab) = f(a)f(b) \quad \text{(当$gcd(a,b) = 1$时)}
					$$
					如果连互质都不需要,那么我们就叫它完全积性函数。
				\end{definition}
				
							\pause
				常见的完全积性函数:
				\begin{align}
					f(n) & = n^a (a \geq 1)\\
					f(n) & = 1 
				\end{align}
							\pause
				常见的积性函数:
				$$
					\mu(n), \eta(n)(\text{(约数个数)}), \sigma(n)(\text{约数和}), \varphi(n)
				$$
			\end{frame}
			\begin{frame}
				积性函数有些"生成规则":
				
				如果$f(n)$和$g(n)$是积性的,那么:
				\begin{itemize}
					\item $h(n) = f(n)g(n)$
					\item $h(n) = \sum_{d \mid n}f(d)$
					\item $h(n) = \sum_{d \mid n}f(d)\mu(\frac{n}{d})$
				\end{itemize}
				都是积性的。
				
				\pause
				
				大家应该会线性筛求积性函数吧。
				
			\end{frame}	
		
			\subsection{例题}
			\begin{frame}{我们来做题吧}
				\begin{block}{例一}
					求欧拉函数:$\varphi(n)$的值($n \leq 10^7$)
				\end{block}
			\end{frame}
			\begin{frame}
				\begin{block}{例二}
					给你$n,m(n,m \leq 10^7)$,求在$[1,m]$中与$n$互质的数的个数。
				\end{block}
			\end{frame}
			\begin{frame}
				\begin{block}{例三}
					给你$n,m(n,m \leq 10^7)$,求在$[1,m]$中与$n$互质的数的和, 模$10^9 + 7$。
				\end{block}
				\pause 
				改成$k$次方的和呢?($0 \leq k \leq 1000$)
			\end{frame}
			\begin{frame}
				\begin{block}{例四}
					给你$n,m(n,m \leq 10^7)$,求在$[1,m]$中的每个数与$n$的最大公约数的和。
				\end{block}
			\end{frame}
			\begin{frame}
				\begin{block}{例五}
					给你$n,m(n,m \leq 10^7)$,求满足$1 \leq i \leq n, 1 \leq j \leq m, gcd(i,j) = 1$的二元组$(i,j)$的对数。
				\end{block}
			\end{frame}
			\begin{frame}
				\begin{block}{例六}
					给你$n,m(n,m \leq 10^7)$,所有满足$1 \leq i \leq n, 1 \leq j \leq m, gcd(i,j) = 1$的二元组$(i,j)$对答案的贡献为$ij$,求最终答案(模$10^9+7$)。
				\end{block}
			\end{frame}
			\begin{frame}
				\begin{block}{例七}
					给你$n,m(n,m \leq 10^7)$,所有满足$1 \leq i \leq n, 1 \leq j \leq m$的二元组$(i,j)$对答案的贡献为$gcd(i,j)$,求最终答案(模$10^9+7$)。
				\end{block}
			\end{frame}
			\begin{frame}
				\begin{block}{例八}
					给你$n,m(n,m \leq 10^7)$,所有满足$1 \leq i \leq n, 1 \leq j \leq m$的二元组$(i,j)$对答案的贡献为$\sigma(gcd(i,j))$,求最终答案(模$10^9+7$)。
				\end{block}
			\end{frame}
			\begin{frame}
				\begin{block}{例九}
					给你$n,m(n,m \leq 10^7)$,所有满足$1 \leq i \leq n, 1 \leq j \leq m$的二元组$(i,j)$对答案的贡献为$lcm(i,j)$,求最终答案(模$10^9+7$)。
				\end{block}
			\end{frame}
\end{document}

