\documentclass{beamer}
\usepackage{xeCJK}
\usepackage{listings}
\usepackage{graphicx}

\title{数学第二讲}
\subtitle{离散对数、元根、反演}
\author{丁尧尧}
\institute{上海交通大学}
\date{\today}
\usetheme{PaloAlto}

\begin{document}
	\maketitle
	\begin{frame}{目录}
		\tableofcontents
	\end{frame}
	
	\section{离散对数}
		\begin{frame}
			对于以下问题:
			\begin{definition}[离散对数]
				给定$a$,$b$,$m$,其中$a$与$m$互素,求最小的非负(正)整数$x$,使得:
				$$
				a^x \equiv b \quad (mod \; m)
				$$
				我们称$x$为在模$m$意义下,以$a$为底的$b$的离散对数,记作$ind_ab$。
			\end{definition}
			给出$a,b,m$,(我们假设$a$与$m$互素)我们如何求$x$呢?
		\end{frame}
		\begin{frame}
			有一个叫做大步小步的算法(Baby Step Giant Step)。
			我们假设$x = ic + j$是一个答案(其中$c$是我们自己选定的一个在$2,m-1$之间的数)。
			我们先计算出:
			$$
			a^1, a^2, a^3, \cdots, a^{b-1}
			$$			
			如果发现其中某个值是$b$,那么我们就找到答案了。否则,我们将这些数放在一个数据结构中(平衡二叉树,哈希表都可以),要求可以通过$a^i$的值快速得到$i$。那么我们再依次算出:
			$$
			ba^{-c}, ba^{-2c}, \cdots, ba^{-kc}
			$$
			每算完一个$ba^{-ic}$,我们就看上面的数据结构中是否有一个值$a^j$等于它,如果有,那么它们满足:
			$$
			a^j \equiv ba^{-ic} \quad (mod \; m)
			$$
			即:
			$$
			a^{ic + j} \equiv b \quad (mod \; m)
			$$
		\end{frame}
		\begin{frame}
			分析复杂度,如果我们上面用哈希表存,那么可以$O(1)$判断某个值是否存在。
			
			
		\end{frame}
	\section{元根}
	\section{反演}
\end{document}

