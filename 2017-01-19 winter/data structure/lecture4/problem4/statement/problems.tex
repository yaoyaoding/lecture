\documentclass[11pt,a4paper,oneside]{article}
\usepackage[english]{babel}
\usepackage{olymp}
\usepackage[dvips]{graphicx}
\usepackage{color}
\usepackage{colortbl}
%\usepackage{expdlist}
%\usepackage{mfpic}
%\usepackage{comment}
\usepackage{multirow}

\usepackage{xeCJK}

%\setCJKmainfont[BoldFont={Hei}]
%{SimSun}
%\setCJKmonofont{FangSong}

\renewcommand{\contestname}{
No.7 High School Winter Training - Data Structure 4 \\
idy002, \today
}    

\begin{document}

\begin{problem}{dcplca}{dcplca.in}{dcplca.out}{1 second}
	
	这是一道练习题,要求你们用链剖来写lca,熟悉链剖的过程。(以节点$1$为根)
	
	\InputFile
	
	第$1$行,一个整数$N$,表示树的节点个数。
	
	接下来$N - 1$行,每行两个数:$u, v$,表示一条边
	
	接下来$1$行,一个整数$Q$,表示询问数。
	
	接下来$Q$行,每行两个整数,$u, v$,表示询问节点$u$和$v$的lca。
	
	\OutputFile
	
	对于每个询问,输出lca的节点编号。
	
	\Example
	
	\begin{example}
		\exmp{
			5
			1 2
			1 3
			2 4
			2 5
			2
			4 5
			3 5
		}{
			2
			1		
	}%
\end{example}

\Note

\begin{itemize}
	\item 对于$100\%$的数据,$1 \leq N, Q \leq 10^5$,$1 \leq u, v \leq N$
\end{itemize}
注意:内存限制为$32MB$。

\end{problem}

\begin{problem}{treekth}{treekth.in}{treekth.out}{3 second} 
	
	给你棵带点权树,一些询问:
	
	\begin{itemize}
		\item \texttt{query u v} 询问节点$u$和$v$之间的简单路径上的点权的中位数\footnote{排序后第第$\lfloor \frac{1 + len}{2} \rfloor$个数,其中$len$是路径的点数}。
	\end{itemize}
	
    \InputFile
    
    第$1$行,一个整数$N$,表示树的点数。
    
    第$2$行,有$N$个整数:$a_1,a_2,\dots,a_n$表示每个点的点权。
    
    接下来$N-1$行,每行两个整数$u,v$表示一条边。
    
    接下来$1$行,一个整数$Q$,表示询问数。
    
    接下来$Q$行,每行两个整数$u,v$,表示一个询问。

    \OutputFile

	对于每个询问,输出中位数。

    \Example

    \begin{example}
        \exmp{
		4
		1 2 3 4
		1 2
		2 3
		3 4
		1
		1 4
        }{
		2
        }%
    \end{example}

    \Note
    
    \begin{itemize}
		\item 对于$30\%$的数据,$1 \leq N, Q \leq 10^3$
		\item 对于$60\%$的数据,$1 \leq N, Q \leq 10^4$
		\item 对于$100\%$的数据,$1 \leq N, Q \leq 10^5$,$1 \leq a_i \leq N$,$1 \leq u, v \leq N$
    \end{itemize}

\end{problem}


\begin{problem}{full}{full.in}{full.out}{2 second} 
	
	我们来个完全版如何?
	
	给你棵带点权的树(以$1$为根),要你完成一些操作。
	
	\begin{itemize}
		\item \texttt{msub u x}:将$u$代表的子树的点权整体加$x$
		\item \texttt{mpth u v x}:将$u$到$v$的简单路径的点权整体加$x$
		\item \texttt{qsub u}:询问子树$u$的点权和
		\item \texttt{qpth u v}:询问路径$u$到$v$的点权和
	\end{itemize}
	
	\InputFile
	
	第$1$行,一个整数$N$,表示树的大小。
	
	第$2$行,$N$个整数:$a_1,a_2,\dots,a_n$表示初始点权
	
	接下来$N - 1$行,每行两个整数:$u, v$表示一条边
	
	接下来$1$行,一个整数$Q$表示操作数。
	
	接下来$Q$行,每行是上面四个操作之一。
	
	\OutputFile
	
	对于每个询问,输出其对应结果。
	
	\Example
	
	\begin{example}
		\exmp{
			5
			1 2 3 4 5
			1 2
			1 3
			2 4
			2 5
			4
			msub 1 1
			mpth 2 5 1
			qsub 1
			qpth 3 5 
		}{
			22
			17
		}%
\end{example}

\Note

\begin{itemize}
	\item 对于$30\%$的数据,$1 \leq N, Q \leq 10^3$
	\item 对于$100\%$的数据,$1 \leq N, Q \leq 10^5$,$1 \leq u, v \leq N$,$1 \leq a_i, x \leq 10^5$
\end{itemize}

\end{problem}

\end{document}
