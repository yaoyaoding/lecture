\documentclass[a4paper,10pt]{article}
%\documentclass[a4paper,10pt]{scrartcl}
\usepackage{multirow}
\usepackage{amsmath}
\usepackage{amsfonts}
\usepackage{amssymb}
\usepackage{xeCJK}
\usepackage[utf8]{inputenc}

\title{题解}
\date{\today}


\begin{document}
	\maketitle
	\section{chess}
	     将棋盘二分染色后将会互相攻击的位置连边,最后就是求二分图的最大独立集,以及方案构造。
	     
	     最大独立集大小 = 点数 - 最大匹配数
	     
	     如果设S是最大独立集,T是点的最小支配集,可以证明S和T刚好互补。
	     
	     构造过程:
	     
	     1. 先把所有未匹配点加入S;
	     
	     2. 如果u在S中,那么所有和u相连的点都在T中;
	     
	     3. 如果u在T中,那么u的匹配点在S中;
	     
	     4. 最后,如果还有点没在两个集合任意一个中,那么一定是左边的点和右边的点个数相同,且互为匹配点,把一边加入S,一边加入T即可。
	     
	   
	\section{tower}
		首先假设知道了最后的排列顺序,我们可以先把这些塔尽量挤在一起,最后再插空。设最小总长度为 x,那么插空的方案数就是 $C(L-x+n,n)$。现在要求出 $F[x]$表示合法的排列数。我们可以从高到低确定每个塔的位置,设 $F[i][k][x]$表示确定到第 $i$ 座塔,还有 $k$ 个区间没有固定端点,最小长度为 $ x $,那么转移相当于将 $ i $ 插入一个未确定区间中,分左右区间是否确定来转移即可。求出 $ F $ 的复杂度为 $ O(n^4) $。最后需要求组合数,因为 $ n<=100,C(i,j)=C(i-1,j-1)+C(i-1,j) $,我们可以用矩阵快速幂先求出 $ C(L-n^2+n,n) $,然后递推求出剩下的,这里复杂度为 $ O(n^3logL) $,总复杂度就是 $ O(n^4) $。
	
	\section{stream}
		首先最大流就是最小割,对于$ M=N-1 $ 的树,很显然只需要割掉两点间最短
		的边;$ N,M $ 很小时可以直接建图暴力。
		
		这样就拿到了前30\%的部分分。
		
		观察发现,我们或是断掉两点间所有简单路径都包含的任何一条边、或是
		断掉不同在两者间任意简单路径上但在一个环内的两条边。
		
		
		我们考虑用线段树和树链剖分来解决。
		
		将每个环断成链后接起来用一棵线
		段树来维护,再将环缩点。在得到的树中,把某一树点u 的权值,设为以u 的
		最顶端结点为S、u 的父树点的最顶端结点为T 的最小割(在刚刚建出的线段树
		中查询)。
		
		继而我们做树链剖分,对于一次询问(S, T),我们求出其LCA,分别考虑两
		条深度递减的路径,并特判(然后直接在线段树中查询)一些特殊的情况,然
		后重链直接在树链剖分结构内查询、轻边在一开始的线段树内查询,取最小值
		即可;对于修改,若修改的是非环边,直接在树剖结构内单点修改,否则我们
		先在一开始建出的线段树中修改,再修改树剖结构内该环对应的树点的重儿子
		的权值。
		
		一道简单的静态仙人掌(一点不在多个环内)......复杂度$ O(M +Qlog 2 N) $。
	
\end{document}
