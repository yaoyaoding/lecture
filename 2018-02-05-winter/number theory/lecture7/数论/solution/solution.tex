\documentclass{article}

\usepackage{amsmath}
\usepackage{amsfonts}
\usepackage{amssymb}

\usepackage{xeCJK}

\begin{document}
	\section{equation}
		\subsection{30\%数据}
			暴力,这时答案本身很小,所以只需要写一个dfs。
		\subsection{70\%数据}
			这个问题等价于“将$m$个相同的球放进$n$个不同的盒子中”的方案数,所以:
			
			$$
				ans \; = \; \binom{n+m-1}{m}
			$$
			
			(记得课上讲的夹棍法吗?)
			
			后者可以用递推公式算:
			$$
				\binom{n}{m} = \binom{n-1}{m-1} + \binom{n-1}{m}
			$$
			
			也可以暴力乘+逆元。
			$$
				\binom{n}{m} = \frac{n!}{m!(n-m)!}
			$$
			
			复杂度后者好点(因为模数不同,前者每次都要推一遍)
		\subsection{余下30\%数据}
			
			用公式推应该会超时,只能用后面那个,求逆元可以用扩展欧几里得,也可以欧拉定理(注意题目中$\varphi(pq) = (p-1)(q-1)$)。当然也可以在在模$p$和模$q$时答案求出来,再用中国剩余定理合并。
			
		\newpage
		
	\section{power}
		
		\subsection{10\%数据}
			暴力循环
			
		\subsection{30\%数据}
			普通二进制快速幂
			
		\subsection{70\%数据}
			高精度 + 二进制快速幂
			
		\subsection{100\%数据}
			十进制快速幂,类比二进制,从低到高维护好$3^{10^{i}}$。
		
		\newpage
		
	\section{comb}
		\subsection{30\%数据}
			暴力算出组合数,用递推公式
			
		\subsection{100\%数据}
			题目就是求:
			$$
				p \mid \binom{n}{i}
			$$
			即:
			$$
				\binom{n}{i} \equiv 0 \; ( mod \; p )
			$$
			的$i$的个数。想到可能和Lucas有关,我们先把问题转化成求:
			$$
				p \nmid \binom{n}{i}
			$$
			的$i$的个数。
			由Lucas定理:
			$$
				\binom{n}{m} \equiv \binom{n_1}{m_1}\binom{n_2}{m_2}\cdots\binom{n_s}{m_s}  \; ( mod \; p )
			$$
			其中$n_1n_2\dots n_s$和$m_1m_2\dots m_s$是$n$和$m$的$p$进制分解。
			容易发现:只要右边每一项模意义下都非$0$,那么右边乘起来都非零,这样的数$m$有:
			$$
				(n_1+1)(n_2+1)\cdots(n_s+1)
			$$
			个,这就是不满足条件的数的个数,那么答案就是:
			$$
				(n + 1) - (n_1+1)(n_2+1)\dots(n_s+1)
			$$
			\newpage
	
\end{document}