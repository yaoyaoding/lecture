\documentclass[11pt,a4paper,oneside]{article}
\usepackage[english]{babel}
\usepackage{olymp}
\usepackage[dvips]{graphicx}
\usepackage{color}
\usepackage{colortbl}
%\usepackage{expdlist}
%\usepackage{mfpic}
%\usepackage{comment}
\usepackage{multirow}

\usepackage{xeCJK}

%\setCJKmainfont[BoldFont={Hei}]
%{SimSun}
%\setCJKmonofont{FangSong}

\renewcommand{\contestname}{
No.7 High School Winter Training - Data Structure 3 \\
idy002, \today
}    

\begin{document}
	\begin{problem}{setmod}{setmod.in}{setmod.out}{2 seconds}{256}
		
		给你一个序列:$a_1 \; a_2 \; a_3 \dots  a_n $,有$m$个操作,操作如下:
		
		\begin{itemize}
			\item \texttt{modify l r x} 将区间$[l,r]$中的每个数修改为$x$
			\item \texttt{change l r x} 将区间$[l,r]$中的每个数加上$x$
			\item \texttt{query l r } 询问区间$[l,r]$中的和
		\end{itemize}
		
		\InputFile
		
		第$1$行$2$个整数:$n \; m$,表示序列长的和操作数.
		
		第$2$行$n$个整数:$a_1 \; a_2 \; a_3 \dots a_n$,表示初始序列.
		
		接下来$m$行,每行是上面三种操作中的一种.
		
		\OutputFile
		
		对于每个询问操作,输出其结果.
		
		\Example
		
		\begin{example}
			\exmp{
				3 3
				1 2 3
				change 1 3 2
				modify 3 3 3
				query 1 3
			}{
			10
		}%
	\end{example}
	
	\Note
	\begin{itemize}
		\item 对于$30\%$的数据,$1 \leq n, m \leq 10^3$
		\item 对于$100\%$的数据,$1 \leq n, m \leq 10^5$,$1 \leq a_i, x \leq n$,$1 \leq l \leq r \leq n$
	\end{itemize}
	
\end{problem}
\begin{problem}{area}{area.in}{area.out}{1 second}
	
	这里是一个格子的世界,我们给出$N$个矩形,要你求他们的面积并。
	
	\InputFile
	
	第$1$行,一个整数$N$。
	接下来$N$行,每行四个整数:$x1,y1,x2,y2$表示矩形$(x_1,y_1)-(x_2,y_2)$。\footnote{这里是离散的情况,意思是当$x_1$与$x_2$相同时,认为有高度还是$1$,可以参考样例}
	
	\OutputFile
	
	输出面积并。
	
	\Example
	
	\begin{example}
		\exmp{
			2
			1 1 2 2
			2 2 3 3
		}{
		7
	}%
\end{example}

\Note
\begin{itemize}
	\item 对于$30\%$的数据,$1 \leq x_1 \leq x_2 \leq 10^3$, $1 \leq y_1 \leq y_2 \leq 10^3$
	\item 对于$100\%$的数据,$1 \leq x_1 \leq x_2 \leq 10^9$,$1 \leq y_1 \leq y_2 \leq 10^9$,$1 \leq N \leq 10^3$
\end{itemize}
\end{problem}

\begin{problem}{array}{array.in}{array.out}{1 second} 
	
	给出一个长度为$N$的数组,每次有三种操作:
	
	\begin{enumerate}
		\item \texttt{modify i x}:把位置$i$的值赋值成$x$。
		\item \texttt{query i}:询问第$i$个位置的值。
		\item \texttt{back t}:把数组变成进行了第$t$个操作之后的版本。
	\end{enumerate}
	
    \InputFile
    
    第$1$行,包含一个整数$N$。
    
    第$2$行,包含$N$个整数:$a_1,a_2,\dots,a_n$。
    
    第$3$行,包含一个整数$Q$,表示操作数目。
    
    接下来$Q$行,每行包含上述三个操作之一。
    
    \OutputFile

	对于每个询问操作,输出其结果。

    \Example

    \begin{example}
        \exmp{
			4
			1 2 3 4
			5
			query 2
			modify 2 3
			query 2
			back 0
			query 2
        }{
			2
			3
			2
        }%
    \end{example}

   \Note
    \begin{itemize}
		\item 对于$30\%$的数据,$1 \leq N, Q \leq 10^3$
		\item 对于$100\%$的数据,$1 \leq N, Q \leq 10^5$,$0 \leq t \leq Q-1$,$1 \leq i \leq N$,$1 \leq x, a_i\leq N$
    \end{itemize}

\end{problem}

\end{document}
