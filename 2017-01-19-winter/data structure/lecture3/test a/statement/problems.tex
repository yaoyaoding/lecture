\documentclass[11pt,a4paper,oneside]{article}
\usepackage[english]{babel}
\usepackage{olymp}
\usepackage[dvips]{graphicx}
\usepackage{color}
\usepackage{colortbl}
%\usepackage{expdlist}
%\usepackage{mfpic}
%\usepackage{comment}
\usepackage{multirow}

\usepackage{xeCJK}

%\setCJKmainfont[BoldFont={Hei}]
%{SimSun}
%\setCJKmonofont{FangSong}

\renewcommand{\contestname}{
No.7 High School Winter Training - Data Structure 3 \\
idy002, \today
}    

\begin{document}
\begin{problem}{distance}{distance.in}{distance.out}{1 second} 
		
		小庆住在一个很特别的国度里,它有$N$个城市,并且只建了$N-1$条双向路,但神奇的是任意两个城市都可以通过这些路连接起来。小庆最近在研究寒假的旅游计划,有时她想快速地知道两个城市之间的距离,于是找你来帮帮解决。
		
		\InputFile
		
		第$1$行一个整数$N$。	
		
		接下来$N-1$行,每行三个整数$u, v, w$,表示城市$u$和城市$v$之间有一条长为$w$的路。
		
		接下来$1$行,包含一个整数$Q$,表示小庆有$Q$个询问。
		
		接下来$Q$行,每行两个整数$u,v$,表示小庆想知道$u$和$v$这两个城市之间的距离。
		
		\OutputFile
		
		对于小庆的每个询问,输出两个城市之间的距离。
		
		\Example
		
		\begin{example}
			\exmp{
				4
				1 2 3
				1 3 4
				2 4 2
				3
				1 1
				1 4
				2 3
			}{
			0
			5
			7
		}
	\end{example}
	
	\Note
	
	\begin{itemize}
		\item 对于$30\%$的数据,$1 \leq N, Q \leq 10^3$
		\item 对于$100\%$的数据,$1 \leq N, Q \leq 10^5$, $1 \leq w \leq 10^3$, $1 \leq u, v \leq N$ 
	\end{itemize}
	
\end{problem}

\begin{problem}{redpacket}{redpacket.in}{redpacket.out}{1 second} 
	(承上题)小漫是小庆那个国家的国王,她住在$1$号城市,$u$号城市如果到$1$必定经过$v$号城市,我们则称$v$号城市管辖$u$号城市($v$号城市也管辖自己)。过年了,小漫想给国家的一些城市发红包,每次她会给$u$号城市管辖的每个城市发放$w$的红包,有时,她也想知道某个城市或被某个城市管辖的城市一共得了多少红包。如下:
	\begin{itemize}
		\item \texttt{give u w} :表示将$u$号城市管辖的每个城市发$w$的红包。
		\item \texttt{single u} :表示询问$u$号城市得了多少红包。
		\item \texttt{all u} :表示询问$u$号城市管辖的城市一共得了多少红包。
	\end{itemize}
	
	\InputFile
	
	第$1$行一个整数$N$。	
	
	接下来$N-1$行,每行三个整数$u, v$,表示城市$u$和城市$v$之间有一条路。
	
	接下来$1$行,包含一个整数$Q$,表示小漫有$Q$个操作。
	
	接下来$Q$行,每行是上面三种操作的一种。
	
	\OutputFile
	
	对每个询问,输出其答案。
	
	\Example
	
	\begin{example}
		\exmp{
			5
			1 2
			1 3
			2 4
			2 5
			3
			give 1 5
			single 2
			all 2
		}{
			5
			15
		}%
\end{example}

\Note

\begin{itemize}
	\item 对于$30\%$的数据,$1 \leq N, Q \leq 10^3$
	\item 对于$100\%$的数据,$1 \leq N, Q \leq 10^5$, $1 \leq u \leq N$, $1 \leq w \leq 10^5$
\end{itemize}

\end{problem}
\begin{problem}{kth}{kth.in}{kth.out}{1 second}
	
	小敏有个可重集$S$,一开始就包含一些整数,现在有三种操作需要你执行:
	
	\begin{itemize}
		\item \texttt{add x} 将整数数$x$加到集合中
		\item \texttt{del x} 如果集合中有整数$x$,则删除一个$x$,否则忽略本操作。
		\item \texttt{query k} 询问这个集合中第$k$小的整数数是多少\footnote{第$k$小指将可重集从小到大排序后,第$k$个位置的数}
	\end{itemize}
	
	\InputFile
	
	第$1$行,两个整数$lbound\;rbound$,分别表示可能出现在集合中的整数的下界和上界。
	
	第$2$行,有$rbound - lbound + 1$个整数,分别为:$a_{lbound},a_{lbound+1},\dots,a_{rbound}$,其中$a_x$表示一开始整数$x$在可重集$S$中的数量。
	
	第$3$行,一个整数$Q$表示操作数。
	
	接下来$Q$行,每行是上面三种操作的一种。
	
	\OutputFile
	
	对于每个询问,输出其结果。
	
	\Example
	
	\begin{example}
		\exmp{
			-2 2
			3 1 5 1 4
			3
			query 4
			del -1
			queery 4
		}{
			-1
			0
		}%
\end{example}

\Note
\begin{itemize}
	\item 对于$30\%$的数据,$1 \leq rbound - lbound + 1 \leq 10^3$,$1 \leq Q \leq 10^3$
	\item 对于$100\%$的数据,$1 \leq rbound - lbound + 1 \leq 10^5$,$1 \leq Q \leq 10^5$,$-10^5 \leq lbound \leq rbound \leq 10^5$, $ lbound \leq x \leq rbound$,$ 0 \leq a_i \leq 10$,数据保证询问合法
\end{itemize}
\end{problem}

\end{document}
